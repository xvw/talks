\documentclass{beamer}

\usepackage{amsmath}
\usepackage{mathtools}
\mathtoolsset{showonlyrefs}
\usepackage[frenchb]{babel}
\usepackage[T1]{fontenc}
\usepackage[utf8]{inputenc}
\usetheme{metropolis}
\usefonttheme{professionalfonts} % required for mathspec
\usepackage{mathspec}
\usepackage{appendixnumberbeamer}
\usepackage{booktabs}
\usepackage[scale=2]{ccicons}
\usepackage{pgfplots}
\usepgfplotslibrary{dateplot}
\usepackage{xspace}
\usepackage{tikz}
\usepackage{graphicx,xcolor}
\usepackage{changepage}
\usepackage{listings}
\usepackage{multicol}

\newcommand{\adjustimg}{% Horizontal adjustment of image
  \checkoddpage%
  \ifoddpage\hspace*{
		\dimexpr\evensidemargin-\oddsidemargin}\else\hspace*{
			-\dimexpr\evensidemargin-\oddsidemargin}\fi%
}


\newcommand{\centerimg}[2][width=\textwidth]{% Center an image
  \makebox[\textwidth]{\adjustimg\includegraphics[#1]{#2}}%
}

\title[Programmation fonctionnelle]{Programmation fonctionnelle,\newline introduction}
\author{Xavier Van de Woestyne}
\date{5 Juillet 2016}
\institute{Dernier Cri}
\setbeamerfont{section in toc}{size=\small}
\setbeamerfont{subsection in toc}{size=\tiny}
\setbeamerfont{subsubsection in toc}{size=\tiny}

\begin{document}



	\maketitle


	\begin{frame}{Bonjour !}
		\begin{itemize}
			\item \alert{@vdwxv} sur Twitter, \alert{@xvw} sur Github ;
			\item développeur à \alert{Dernier Cri} ;
			\item travail de recherche qui n'avance pas ;
			\item de 2000 à 2008 : \textbf{PHP} (et d'autres langages impératifs) ;
			\item depuis 2008 : \textbf{Ruby}, \alert{Erlang}, \alert{Haskell}, \alert{OCaml/Fsharp} ;
			\item Meetup \alert{LilleFP}
		\end{itemize}
	\end{frame}

	\begin{frame}{Sommaire}
    \begin{multicols}{2}
		    \tableofcontents
        \addtocontents{toc}{~\vspace{-3\baselineskip}}
    \end{multicols}
	\end{frame}

	\section{Préambule}

	\subsection{Objectifs}
	\begin{frame}{Objectifs}
		\begin{itemize}
			\item Présentation sommaire ;
			\item démystification (je suis nul en math...);
			\item présentation de motifs ;
		\end{itemize}
		Présenter des concepts appropriables dans 80\% des langages
		modernes.
	\end{frame}

  \begin{frame}{Pourquoi ?}
    \begin{itemize}
      \item Passé et futur de la programmation ;
      \item facile à appréhender ;
      \item OCaml, Haskell, FSharp, Scala, Clojure, Ruby, Pytho, JavaScript, C++, CSharp, Smalltalk, Java ;
      \item Elixir, Elm ;
      \item FRP, continuations/promesses, parallélisme ;
      \item découverte.
    \end{itemize}
  \end{frame}

  \begin{frame}
    Tout commence avec ...
  \end{frame}

	\subsection{Mise en contexte historique}
	\subsubsection{Logique formelle et décision}
	\begin{frame}{Le problème de la décision}
		Comment définir l'ensemble des expressions calculables?
		\begin{quote}
			En logique formelle,
			une propriété mathématique est dite décidable s'il existe un procédé
			mécanique qui détermine, au bout d'un temps fini, si elle est vraie ou
			fausse dans n'importe quel contexte possible.
		\end{quote}
	\end{frame}

	\begin{frame}
		\centerimg[width=\textwidth, height=250pt]{figures/logic.pdf}
	\end{frame}

	\begin{frame}{Formalismes pour représenter \newline l'intégralité des fonctions calculables}
		Trois réponses pour le prix d'une ...
		\begin{itemize}
			\item \textbf{1935} : \alert{le lambda-calcul} de Church ;
			\item \textbf{1935} : les fonctions récursives de Gödel et Kleen ;
			\item \textbf{1936} : les machines de Turing.
		\end{itemize}
		C'est une découverte (et non une invention).
	\end{frame}

  \begin{frame}{Types et preuves}
    Les systèmes formels servant la logique sont étroitement liés aux preuves,
    donc, par extension, aux systèmes de types :
    \begin{itemize}
      \item Lutter contre le paradoxe de la logique formelle (ex: Paradoxe du Barbier) ;
      \item Correspondance preuve-programme (l'isomorphisme de Curry-Howard) : logique mathématique <-> Informatique théorique ;
      \item Hindley-Milner, Martin Löf.
    \end{itemize}
  \end{frame}

	\subsubsection{Le $\lambda$-Calcul}
	\begin{frame}{Le $\lambda$-Calcul, la base}
		\begin{itemize}
			\item Un langage théorique ;
			\item la base de tout langage fonctionnel ;
			\item initialement non typé, mais peut l'être (Système F).
		\end{itemize}
		Seulement trois éléments gramaticaux
		\begin{itemize}
			\item \textbf{Les variables} : x, y, ..., des lambda-termes ;
			\item \textbf{les applications} :  f x, est un lambda-terme si f et x sont des lambda-termes ;
			\item \textbf{les abstractions} :  $\lambda x.v$ est un lambda-terme si x est une variable et v un lambda-terme.
		\end{itemize}
	\end{frame}

	\begin{frame}{Quelques constructions}

		\begin{columns}[t]
			\begin{column}{5cm}
				\textbf{Les entiers de Church :}
			\end{column}
			\begin{column}{5cm}
				\textbf{Les Boolééns :}
			\end{column}

		\end{columns}

		\begin{columns}[t]
 			\begin{column}{5cm}

				\begin{gather}
					0 = \lambda fx.x \\
					1 = \lambda fx.f \; x \\
					2 = \lambda fx.f \; (f \; x) \\
					3 = \lambda fx.f \; (f \; (f \; x)) \\
				\end{gather}

 			\end{column}

 			\begin{column}{5cm}

				\begin{gather}
					true = \lambda ab.a \\
					false = \lambda ab.b \\
					or = \lambda pq.p\;p\;q \\
					and = \lambda pq.p\;q\;p
				\end{gather}

 			\end{column}
		\end{columns}
	\end{frame}

	\begin{frame}{Approcher le \og pur \fg orienté Objets}
		\begin{columns}[t]
			\begin{column}{5cm}
				\lstinputlisting[language=Ruby, basicstyle=\tiny]{src/integer.rb}
			\end{column}
			\begin{column}{5cm}
				\lstinputlisting[language=Ruby, basicstyle=\tiny]{src/bool.rb}
			\end{column}

		\end{columns}
		Seul l'\alert{IO} ne semble pas représentable en pur objet.
	\end{frame}

	\section{Programmation fonctionnelle}
	\subsection{Première définition}
	\begin{frame}{Un langage fonctionnel}
    Un langage fonctionnel est \alert{simplement} un langage qui peut
    manipuler des fonctions comme des valeurs (lambdas).
    \begin{description}
      \item[1935] $\lambda$-calcul ;
      \item[1958] Lisp (premier langage moderne !)
    \end{description}
    Un langage fonctionnel peut être
    \begin{itemize}
      \item Pur ou impur;
      \item strict ou non strict*.
    \end{itemize}
    * Les paramètres doivent être évalué complètement avant d'être appelés.
	\end{frame}

  \subsection{Les fonctions}
  \begin{frame}{Concept principal : les fonctions}

    \centerimg[width=300pt, height=100pt]{figures/function.pdf}

  \end{frame}

  \begin{frame}
    \lstinputlisting{src/fun.ml}
  \end{frame}

  \begin{frame}{A propos des fonctions}
    \begin{itemize}
      \item Les fonctions sont des valeurs ;
      \item on peut en utiliser partout !
      \item Les fonctions ne \textbf{prennent qu'un seul argument} (héritage du $\lambda$-calcul).
    \end{itemize}
  \end{frame}

  \begin{frame}
    \lstinputlisting{src/argument.ml}
  \end{frame}

  \subsubsection{Curryfication}
  \begin{frame}{Curryfication}
    En $\lambda$-calcul, les n-uplet n'existent pas encore. On utilise la logique combinatoire de
    Haskell Curry et Moses Schönfinkel pour utiliser plusieurs arguments :
    \begin{itemize}
      \item $let \; add \; x \; y \; = \; x + y$
      \item $let \; add \; = \; fun \; x \; y \; \rightarrow \; x + y$
      \item $let \; add \; = \; fun \; x \; \rightarrow \; (fun \; y  \; \rightarrow \; x + y)$
    \end{itemize}
    Les fonctions peuvent être des valeurs d'entrée, mais aussi des valeurs de retour.
  \end{frame}

  \begin{frame}{Conséquence (cool) : l'application partielle}
    \lstinputlisting{src/curry.ml}
    \alert{add 1} renvoie une fonction qui attend le paramètre $y$
  \end{frame}

  \subsubsection{Expressivité}
  \begin{frame}{Expressivité : moins de dépendance à la grammaire}
    \lstinputlisting[language=php]{src/foreach.php}
  \end{frame}

  \begin{frame}{Expressivité : moins de dépendance à la grammaire}
    \lstinputlisting{src/foreach.ml}
  \end{frame}

  \subsubsection{Compositions}
  \begin{frame}{Les fonctions sont des légos...}
    Au sens figuré :)\newline L'enjeu du programmeur fonctionnel est de combiner des fonctions
    entre elles pour construire des fonctions plus complexes.
    \begin{itemize}
      \item une action atomique est une fonction ;
      \item une collection d'action atomique est un service ;
      \item une collection de services est une application.
    \end{itemize}
    Ce qui permet de reproduire une grande partie des motifs de conceptions
    de l'orienté objet...
  \end{frame}

  \section{Les types comme outil de design}
  \begin{frame}{Un système de type}
    \begin{itemize}
      \item Une extension à la théorie des ensembles ;
      \item donne une notion de domaines aux fonctions ;
      \item garantit une certaines sûrté ;
      \item peut inférer un type pour une valeur ;
      \item peut être un outil de design précis.
    \end{itemize}
  \end{frame}

  \subsection{Types algébriques}

  \begin{frame}{Types algébriques}

    \begin{columns}[t]
      \begin{column}{5cm}
        \textbf{Types primitifs :}
        \begin{itemize}
          \item int ;
          \item float ;
          \item char ;
          \item bool ;
          \item string etc.
        \end{itemize}
      \end{column}
      \begin{column}{5cm}
        \textbf{Composition de types :}
          \lstinputlisting{src/ty.ml}
      \end{column}
    \end{columns}

  \end{frame}

	\begin{frame}{Plus de types}
    \begin{columns}[t]
      \begin{column}{4cm}
        \textbf{Types paramétrés :}
        \lstinputlisting{src/option.ml}
        \begin{itemize}
          \item $\alpha$ list ;
          \item $(\alpha, \beta)$ Hashtbl.t
        \end{itemize}
      \end{column}
      \begin{column}{6cm}
        \textbf{Types récursifs :}
          \lstinputlisting{src/rec.ml}
        \textbf{Alias de types :}\newline
        type prenom = string \newline
        \textbf{Types abstraits}
      \end{column}
    \end{columns}
  \end{frame}

  \begin{frame}{Type des fonctions}
    \begin{itemize}
      \item x = 'h' : $char$ ;
      \item id x : $\alpha \rightarrow \alpha$ ;
      \item add x y : $int \rightarrow int \rightarrow int$ ;
      \item couple x y = (x, y) : $\alpha \rightarrow \beta \rightarrow (\alpha * \beta)$ ;
      \item map f list = $(\alpha \rightarrow \beta) \rightarrow \alpha \; list \rightarrow \beta \; list$.
    \end{itemize}
  \end{frame}

  \begin{frame}{Déconstruction de types}
    On peut déconstruire des types composés au moyen du \alert{Pattern-Matching} :
    \lstinputlisting{src/dec.ml}
  \end{frame}

  \subsubsection{Apport des types}
  \begin{frame}{Apport des types}
    \begin{itemize}
      \item \alert{Sécurité} du programme (vérification statique !) ;
      \item documentation : $map = (\alpha \rightarrow \beta) \rightarrow \alpha \; list \rightarrow \beta \; list$ ;
      \item Domain-Driven-Design : Types nommés/alias de types, options etc.
      \item Possibilité de faire découler des règles \alert{simples} au moyen d'assertions \alert{simples}.
     \end{itemize}
  \end{frame}

  \section{Propriété par le type : le monoïde}

  \begin{frame}{Le monoïde}
    Tout type $\alpha$ qui possède un opérateur de composition ($\oplus$) et un élément neutre ($e$) et qui est régit par ces trois règles :
    \begin{itemize}
      \item $\oplus : \alpha \rightarrow \alpha \rightarrow \alpha$ (la cloture)
      \item $(x \oplus y) \oplus z = x \oplus (y \oplus z)$ (l'associativité)
      \item $e \oplus x = x$ et $x \oplus e = x$ (l'élément neutre)
    \end{itemize}
    est un monoïde.
  \end{frame}

  \begin{frame}{Par exemple, les entiers}
    \begin{itemize}
      \item $\oplus = (+)$
      \item $1+(2+3) = (1+2)+3$
      \item $e = 0$
    \end{itemize}
  \end{frame}


  \begin{frame}{Par exemple, les strings}
    \begin{itemize}
      \item $\oplus = String.concat$
      \item $e = \; ''$
    \end{itemize}
  \end{frame}

  \begin{frame}{Par exemple, les listes}
    \begin{itemize}
      \item $\oplus = List.concat$
      \item $e = []$
    \end{itemize}
  \end{frame}

  \begin{frame}{Par exemple, les endofonctions ($\alpha \rightarrow \alpha$)}
    \begin{itemize}
      \item $\oplus = (.) $ ($f.g x= f(g(x))$)
      \item $e = id$
    \end{itemize}
  \end{frame}


    \begin{frame}{Apport des règles : l'opérateur OPLUS}
      Transforme une opération binaire en opération qui fonctionne sur des listes (reduce/fold)
      \newline
      \newline
      $fold \oplus (liste \; de \; \alpha) : \alpha$
    \end{frame}


      \begin{frame}{Apport des règles : l'associativité}
        Accumulation incrémentale, parallélise facilement. Par exemple : \newline1 + 2 + 3 + 4
        \centerimg[width=100pt, height=80pt]{figures/divide.pdf}
        Chaque noeud peut être calculé sur une unité de calcul différente (et parallèle, par exemple). Ou chaque
        résultat peut être accumulé.
      \end{frame}

      \begin{frame}{Apport des règles : l'élément neutre}
        Permet facilement de traiter les cas où des données sont invalides/manquantes.
      \end{frame}

      \begin{frame}{Dériver un type $\alpha$ en monoïde}
        Il suffit d'implémenter l'opérateur $\oplus$, en respectant les règles évoquées. \newline
        Il est possible de décorer un type $\beta$ en un monoïde au moyen d'une fonction de map.
      \end{frame}

      \begin{frame}
        \centerimg[width=150pt, height=220pt]{figures/mapreduce.pdf}
      \end{frame}

      \begin{frame}{Et voici ... Map/reduce de Google}
        Si on occulte la notion de Speed Up. (but no time ...)

      \end{frame}


  \section{Conclusion}
  \subsection{Influence et application}

  \begin{frame}{Influence et application}

    \begin{itemize}
      \item TypeScript, Flow, HaXe
      \item Applicable à des langages non typés (les promesses)
      \item Evolution des langages (Java, CSharp)
      \item révolutionner la navigation dans le web (avec des continuations !)
    \end{itemize}

  \end{frame}

  \subsection{Plus de super-pouvoirs}

  \begin{frame}{Plus de super-pouvoirs}
    \begin{columns}[t]
      \begin{column}{5cm}
        \textbf{Jutsus fonctionnels}
        \begin{itemize}
          \item Monades
          \item Comonades
          \item Flèches
          \item Structure de données purement fonctionnelles
        \end{itemize}
      \end{column}
      \begin{column}{5cm}
        \textbf{Typages}
        \begin{itemize}
          \item Types existentiels
          \item Types fantômes
          \item Types algèbriques généralisés
        \end{itemize}
      \end{column}
    \end{columns}

  \end{frame}


  \subsection{Fin}
  \begin{frame}{Fin}
    Merci, questions, remarques ?
  \end{frame}

\end{document}
